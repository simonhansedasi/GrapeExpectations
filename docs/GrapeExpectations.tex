\documentclass[11pt,a4paper]{article}
\usepackage[utf8]{inputenc}
\usepackage{graphicx}
\usepackage{amsmath}
\usepackage{geometry}
\geometry{margin=0.8in}
% Reduce title spacing
\usepackage{titling}
\setlength{\droptitle}{-2cm}   % Move title up

\title{Grape Expectations 
\begin{center}
    {\large \textit{Precision Vineyard Monitoring With Distributed Fiber-Optic Sensing}}
\end{center}}

\author{Simon-Hans Edasi}
\date{December 12, 2025}


\begin{document}

\maketitle

\section{Problem Statement}

Modern vineyards face an increasingly complex combination of climate volatility, water scarcity, labor limitations, and the rising demand for high-precision, data-driven management. Growers must optimize irrigation, detect stress early, monitor soil–plant interactions, and improve block-level uniformity—yet existing monitoring tools (spot soil probes, weather stations, drone snapshots, manual scouting) provide only fragmented, sparse, and delayed information. Large vineyards, such as those in California’s Central Valley, can span hundreds to thousands of acres, making it impractical to achieve continuous, high-resolution environmental monitoring with traditional instrumentation. The result is inefficiency, inconsistent fruit quality, and suboptimal resource use.

\section{Technology Overview}

Distributed Fiber Sensing (DFS)—which includes Distributed Temperature Sensing (DTS), Distributed Acoustic Sensing (DAS), and Distributed Strain Sensing (DSS)—transforms a simple fiber-optic cable into a continuous, kilometer-scale sensor with meter-scale resolution. A single interrogator unit measures backscattered light along the fiber to infer environmental conditions in real time.

\begin{itemize}
    

\item DTS provides temperature fluctuations along every meter of fiber.

\item DAS detects acoustic and vibration signals over long distances.

\item DSS measures strain, deformation, and infrastructure loading.

\end{itemize}
\noindent
In agricultural settings, temperature and acoustic signals carry meaningful information about plant and soil processes: root-zone water infiltration produces identifiable thermal fronts, transpiration and canopy stress influence local microclimates, and soil heterogeneity generates persistent temperature contrasts. These effects create spatial patterns that DFS systems can detect at meter-scale resolution. By burying, trenching, or placing fiber along vine rows or at the root zone, the vineyard becomes a live, high-resolution sensing network. One interrogator can monitor several fiber runs using an optical switch, enabling scalable coverage.

\section{Proposed Solution}

This project introduces the use of DFS as a new, high-resolution monitoring strategy for viticulture. The goal is to produce a continuous environmental map of the vineyard, enabling growers to see temperature gradients, irrigation fronts, vine water uptake dynamics, frost risk, soil heterogeneity, and block-level anomalies with unprecedented detail. The approach emphasizes:

\begin{itemize}

\item \textbf{Large-Scale Coverage}

\noindent
Fiber can span tens of kilometers, allowing entire vineyard blocks—or multiple blocks—to be monitored with a single measurement system.

\item \textbf{High Spatiotemporal Resolution}

\noindent
Instead of a few isolated probes, growers gain thousands of virtual sensors capturing real-time temperature and moisture-related signatures across the entire landscape.

\item \textbf{Actionable Insights}

\noindent
DFS can identify nonuniform irrigation distribution, detect vine stress patterns, monitor canopy microclimate, and provide early alerts for frost, leaks, or equipment failures.

\item \textbf{Integrative Platform}

\noindent
DFS data can merge with weather, NDVI, soil maps, and management records to build a comprehensive phenotyping and decision-support platform.

\end{itemize}
\subsection*{Phase 1 Pilot Deployment}

To validate this approach, the initial pilot will deploy approximately 1–2 km of shallow-buried fiber across a single 5–10 acre vineyard block. The interrogator will sample temperature at 1~m spatial resolution and sub-minute temporal resolution, enabling the detection of irrigation infiltration dynamics, soil moisture fronts, and microclimatic heterogeneity. This pilot requires only one interrogator and minimal trenching, providing a low-cost, high-impact demonstration of DFS capability in viticulture.


\section{Impact and Benefits}
\begin{itemize}
\item \textbf{Water Efficiency:}

\noindent
Track irrigation infiltration in real time and correct distribution problems immediately.

\item \textbf{Quality Consistency:}

\noindent
Detect microclimatic variations influencing ripening, leading to more uniform fruit.

\item \textbf{Risk Reduction:}

\noindent
Early frost detection, leak detection, and anomaly identification reduce losses.

\item \textbf{Labor Reduction:}

\noindent
Continuous sensing reduces manual scouting and probe maintenance.

\item \textbf{Scalability:}

\noindent
The optical-switch system allows one interrogator to serve many fiber segments, enabling cost-effective monitoring of large producers.
\end{itemize}
\section{Conclusion}

Distributed Fiber Sensing provides a transformative opportunity for modern viticulture, offering the spatial coverage of remote sensing with the precision of in-situ instrumentation. By deploying fiber strategically across vineyard blocks and integrating interrogator measurements into a unified monitoring framework, growers can unlock detailed insights into environmental conditions, resource use, and vineyard performance. This project aims to build the first scalable, field-ready implementation of DFS for viticulture, establishing a new technological foundation for precision agriculture. \\

\noindent
\textbf{Next Steps:} The immediate requirement for moving this project forward is access to a fiber-optic interrogator (DTS or DAS). With hardware in hand, the Phase~1 pilot can be deployed and evaluated within a single growing season, providing the first field-scale demonstration of DFS for viticulture.




\end{document}
